% Options for packages loaded elsewhere
\PassOptionsToPackage{unicode}{hyperref}
\PassOptionsToPackage{hyphens}{url}
%
\documentclass[
  12pt,
]{article}
\usepackage{amsmath,amssymb}
\usepackage{lmodern}
\usepackage{setspace}
\usepackage{iftex}
\ifPDFTeX
  \usepackage[T1]{fontenc}
  \usepackage[utf8]{inputenc}
  \usepackage{textcomp} % provide euro and other symbols
\else % if luatex or xetex
  \usepackage{unicode-math}
  \defaultfontfeatures{Scale=MatchLowercase}
  \defaultfontfeatures[\rmfamily]{Ligatures=TeX,Scale=1}
  \setmainfont[]{Times New Roman}
\fi
% Use upquote if available, for straight quotes in verbatim environments
\IfFileExists{upquote.sty}{\usepackage{upquote}}{}
\IfFileExists{microtype.sty}{% use microtype if available
  \usepackage[]{microtype}
  \UseMicrotypeSet[protrusion]{basicmath} % disable protrusion for tt fonts
}{}
\usepackage{xcolor}
\IfFileExists{xurl.sty}{\usepackage{xurl}}{} % add URL line breaks if available
\IfFileExists{bookmark.sty}{\usepackage{bookmark}}{\usepackage{hyperref}}
\hypersetup{
  hidelinks,
  pdfcreator={LaTeX via pandoc}}
\urlstyle{same} % disable monospaced font for URLs
\usepackage[margin=2.5cm]{geometry}
\usepackage{graphicx}
\makeatletter
\def\maxwidth{\ifdim\Gin@nat@width>\linewidth\linewidth\else\Gin@nat@width\fi}
\def\maxheight{\ifdim\Gin@nat@height>\textheight\textheight\else\Gin@nat@height\fi}
\makeatother
% Scale images if necessary, so that they will not overflow the page
% margins by default, and it is still possible to overwrite the defaults
% using explicit options in \includegraphics[width, height, ...]{}
\setkeys{Gin}{width=\maxwidth,height=\maxheight,keepaspectratio}
% Set default figure placement to htbp
\makeatletter
\def\fps@figure{htbp}
\makeatother
\setlength{\emergencystretch}{3em} % prevent overfull lines
\providecommand{\tightlist}{%
  \setlength{\itemsep}{0pt}\setlength{\parskip}{0pt}}
\setcounter{secnumdepth}{-\maxdimen} % remove section numbering
\usepackage{ragged2e}
\ifLuaTeX
  \usepackage{selnolig}  % disable illegal ligatures
\fi

\author{}
\date{\vspace{-2.5em}}

\begin{document}

\setstretch{1.5}
\begin{titlepage}

\par
\raisebox{-.5\height}{\includegraphics[width=3cm]{logos/Logo_EHESS.jpg}}
\hfill
\raisebox{-.5\height}{\includegraphics[width=5.5cm]{logos/logo_ens.png}}
\par


\vspace{1cm}
\begin{center}

\normalsize{\textit{Master Sciences Sociales - Parcours Quantifier en Sciences Sociales}}

2021-2022

\vspace{5mm}

\textsc{Mémoire de recherche}

\vfill 

{\Large La désinstitutionalisation en maison d’enfant à caractère sociale (MECS) au travers de l’orientation des enfants placés entre les hébergements\par}

\vfill

{\large\itshape Soutenu par}

{\large Élodie Lemaire}

\vspace{5mm}

\textit{Session}

Juin 2022

\vspace{5mm}

\textit{Sous les directions de}

Isabelle Frechon \& Marie Plessz

\vspace{5mm}

\textit{Relectrice}

Laure Hadj

\end{center}
\end{titlepage}
\newpage

\pagenumbering{roman}

\phantomsection
\addcontentsline{toc}{chapter}{Remerciements}
\chapter*{Remerciements}

\vspace{.5cm}

~~~~~~Je tiens à remercier

\newpage

\phantomsection
\addcontentsline{toc}{chapter}{Avant-propos}
\chapter*{\huge Avant-propos}

\vspace{.5cm}

~~~~~~Ce mémoire a été réalisé sur R Markdown, l'intégralité des
analyses sont disponibles sur Github au lien suivant : .

Pour assurer la reproductibilité des analyses, le package Renv a aussi
été employé. Il permet d'effectuer une capture du système
d'exploitation, des mises à jour de R et de RStudio ainsi que de
l'ensemble des packages employés. Ainsi, même si des mises à jours sont
effectuées sur R, RStudio ou n'importe lequel des packages employé
rendant impossible l'analyse, à l'aide de Renv il est possible de
restaurer les versions utilisées pour reproduire les analyses de ce
mémoire.

\newpage

\end{document}
