\usepackage{ragged2e} % alignement
\usepackage{setspace} % espace entre les lignes
\usepackage{xunicode}
\usepackage[french]{babel}

% pour les titres de tables et figures
\usepackage[normal,labelfont=bf]{caption}
\usepackage{subcaption}
\captionsetup{format=hang,font=small}
\captionsetup[table]{name=Table}
\captionsetup[figure]{name=Figure}

% pour s'assurer que les notes de bas de page reste bien en bas
\usepackage[bottom]{footmisc}

% pour les encadrés
\usepackage{mdframed}


% pour la police
\usepackage[utf8]{inputenc}


% pour table des matières
\usepackage{chngcntr}
\counterwithin{figure}{section}
\counterwithin{table}{section}


\usepackage[toc,title,page]{appendix}

% Pour ajouter le titre de chapitre dans lequel on est en haut de la page
\usepackage{fancyhdr} 
\pagestyle{fancy}
\fancyhead[LE,RO]{\thechapter}

\usepackage{amsthm}
\newtheorem{hyp}{Hypothèse}

\usepackage{natbib}