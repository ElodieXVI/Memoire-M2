\usepackage{ragged2e} % alignement
\usepackage{setspace} % espace entre les lignes

% pour les titres de tables et figures
\usepackage[normal,labelfont=bf]{caption}
\usepackage{subcaption}
\captionsetup{format=hang,font=small}
\captionsetup[table]{name=Tableau}
\captionsetup[figure]{name=Graphique}

% pour s'assurer que les notes de bas de page reste bien en bas
\usepackage[bottom]{footmisc}

% pour les encadrés
\usepackage{mdframed}


% pour la police
\usepackage[utf8]{inputenc}
\usepackage[T1]{fontenc}


% pour table des matières
\usepackage{chngcntr}
\counterwithin{figure}{section}
\counterwithin{table}{section}

% pour afficher "Partie" avant chaque titre de section
\usepackage{titlesec}
\titleformat{\section}{\Large\bfseries}{Partie \thesection.}{1em}{}

\usepackage[toc,title,page]{appendix}

% Pour ajouter le titre de chapitre dans lequel on est en haut de la page
\usepackage{fancyhdr} 
\pagestyle{fancy}
\fancyhead[LO,LE]{}

\usepackage{hyperref}
\hypersetup{
    pdftitle = {Élodie Lemaire - Mémoire M2 QESS},
    pdfauthor = {Élodie Lemaire}
}

\usepackage{amsthm}
\newtheorem{hyp}{Hypothèse}